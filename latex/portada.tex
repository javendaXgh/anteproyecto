%\vspace{-2.0cm}
\pagenumbering{roman}

\begin{center}
	UNIVERSIDAD CENTRAL DE VENEZUELA\\
	FACULTAD DE CIENCIAS\\
	POSTGRADO EN CIENCIAS DE LA COMPUTACI\'ON\\

	\begin{figure}
						\centering
						  \includegraphics[height=.7\textwidth]{images/UCV.png}
  \end{figure}
  \vspace{1.5cm}
  \large{\textbf{ANTEPROYECTO:\\ Solución Sistema Complementario Saber UCV (SSCSU): extracción de datos desde PDF, clasificación y construcción de un sistema de recuperación de información en línea}}

  \vspace{3cm}
  Anteproyecto presentado  \\
  por el Econ. José Miguel Avendaño Infante\\
  Tutor: Dr. Andrés Sanoja\\
  \vspace{1.5cm}
  Caracas, Octubre de 2021
\end{center}




%\newpage


\setlength{\abovedisplayskip}{-5pt}
\setlength{\abovedisplayshortskip}{-5pt}
\thispagestyle{empty}

\thispagestyle{empty}

\newpage
\thispagestyle{empty}
\large{\textbf{Resumen:}}

Se presenta la propuesta de un sistema denominado **Solución Sistema Complementario Saber UCV**  para hacer búsquedas de texto completo sobre los resúmenes de las tesis que se encuentran alojadas en el repositorio institucional Saber UCV (www.saber.ucv.ve). 

Se aplican técnicas de Procesamiento de Lenguaje Natural (NLP), de Minería de Texto y de indexación en base de datos sobre los textos y se muestran los resultados con tablas interactivas, visualizaciones con gráficos y gráfos de coocurrencias de palabras.

Esta propuesta se basa principalmente en que mediante la búsqueda de palabras o frases se genere un (*query*) y con la técnica conocida como "*full text search*" se puedan extraer los trabajos en que se encuentran contenidas tales palabras y a partir de ahí enriquecer la experiencia del usuario con la presentación de la información recuperada.

La aplicación se diseña como un sistema distribuido bajo la arquitectura cliente-servidor en distintos contenedores soportando cada uno un proceso para el funcionamiento, teniendo entre los principales el de base de datos, el servidor de la aplicación y otro con los distintos procesamientos que son efectuados sobre los textos.

También se propone un algoritmo que permite clasificar las tesis por el área académica donde cursó estudios el autor del correspondiente trabajo y así resolver el problema  que actualmente presenta  Saber UCV donde no están disponibles estas clasificaciones.


\thispagestyle{empty}

\maketitle


